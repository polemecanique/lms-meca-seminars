% !TeX spellcheck = en_us
\documentclass[a4paper,11pt,fleqn]{article}

\usepackage[left=1.50cm, right=1.50cm, top=4cm, bottom=2cm, 
headheight= 2.5cm, headsep= 4pt, footskip=25pt]{geometry}
\usepackage{fancyhdr}
\usepackage{lmodern}
\usepackage[table]{xcolor}
\usepackage{amssymb,amstext,amsmath}
\usepackage{booktabs}
\usepackage{bm}
\usepackage{setspace}
\usepackage{graphicx}
\usepackage[colorlinks]{hyperref}
\setlength{\parindent}{0mm}
\setlength{\parskip}{2mm}
\usepackage{enumitem}
\usepackage{titlesec}
\usepackage{gensymb}
\usepackage{pgfgantt}
\usepackage{lscape}
\usepackage[absolute]{textpos}
\usepackage{comment}
\usepackage{cite}
\usepackage{tikz}
\usepackage{pgfplots}
\usetikzlibrary{shapes.geometric}
\usepackage{array}
\newcolumntype{P}[1]{>{\centering\arraybackslash}p{#1}}
\usepackage{longtable}
\usepackage[font={small}]{caption}
\usepackage[most]{tcolorbox}
\usepackage{tikz}
\usepackage{stmaryrd}
%\usepackage[none]{hyphenat}
\usepackage{ctable}
\usepackage{ragged2e}
\usepackage[official]{eurosym}
\newcommand*\circled[1]{\tikz[baseline=(char.base)]{
		\node[shape=circle,draw,inner sep=1pt] (char) {#1};}}
\usepackage[symbol]{footmisc}
\renewcommand{\thefootnote}{\fnsymbol{footnote}}
\renewcommand{\thempfootnote}{\fnsymbol{mpfootnote}}
\renewcommand*\footnoterule{}

\usepackage{xcolor}
\definecolor{docColor}{cmyk}{1.00	, 1.00  , 0   , 0.40}  % 0.90\% of black
\color{docColor}

\newtcolorbox{myhlbox}[2][]{colback=white, colframe=docColor, width=1.0\linewidth,
	boxrule=2.0pt, left=0.0pt, right=0.0pt, top=3pt, bottom=0.0pt,
	colbacktitle=docColor!20, coltitle=black, enhanced, attach boxed title to top left={xshift=3mm, yshift=-1mm},
	title=#2, #1}

\hypersetup{
	colorlinks=true,
	citecolor=blue,
	filecolor=magenta,
	linkcolor=blue,
	urlcolor=blue,
	pdftitle={SeminarAnnouncements},
	%bookmarks=true,,
}

\graphicspath{{./figs/}}

\definecolor{dgreen}{rgb}{0.0, 0.5, 0.0}

\allowdisplaybreaks

\newcommand*\mystrut[1]{\vrule width0pt height0pt depth#1\relax}
\renewcommand{\thesection}{\arabic{section}.}
\renewcommand{\thesubsection}{\arabic{section}.\arabic{subsection}}
\renewcommand{\thesubsubsection}{\arabic{section}.\arabic{subsection}.\arabic{subsubsection}}
\titleformat{\section}{\sffamily\large\bfseries}{\thesection}{1.0em}{}
\titleformat{\subsection}{\sffamily\large\bfseries}{\thesubsection}{1.0em}{}
\titleformat{\subsubsection}{\sffamily\normalsize\bfseries}{\thesubsubsection}{1.0em}{}
\titlespacing\section{0pt}{0pt plus 2pt minus 2pt}{0pt plus 2pt minus 2pt}
\renewcommand{\baselinestretch}{1.50}

\input{format/defs}

\fancypagestyle{titlepage}{
	\fancyhf{}
	\insertLXheader
}

\fancypagestyle{plain}{%
	\fancyhf{}
	\renewcommand{\headrulewidth}{0pt}
	\renewcommand{\footrulewidth}{0pt}
	\insertnormalpage
}

\pagestyle{plain}

\setlength\arrayrulewidth{2pt}

\begin{document}
	\sffamily
	\thispagestyle{titlepage}
	\vspace*{-2em}
	\begin{center}
		\huge \textbf{John Kolinski visit itinerary}
	\end{center}
	\begin{center}
		\Large 
		Day 1:~February 26 \\
		\vspace*{0.5em}
		\begin{tabular}{| p{4cm} p{13cm} |}
			\hline
			18:00 & Arrival in Paris\\
			\hline
		\end{tabular}
		
		Day 2:~ February 27 \\
		\vspace*{0.5em}
		\begin{tabular}{| p{4cm} p{13cm} |}
			\hline
			0900 -- 0930 & Arrival on campus (Vignesh) \\
			0945 -- 1030 & Camille Duprat \\
			1030 -- 1115 & Kostas Danas\\
			1115 -- 1200 &  Etienne  \\
			1145 -- 1300 & Lunch (with Vignesh, Etienne, Nick, Camille, Sascha, Kostas) \\
			1330 & Seminar preparation \\
			1400 -- 1500 & \textbf{Seminar (Amphi Becquerel)} \\ 
			1530 -- 1600 & Sascha Hilgenfeldt \\
			1600 -- 1645 &  Marie-Jean Thoraval \\
			1645 -- 1745 & Wladimir and Daniel \\
			1830 -- 2000 & Dinner with Vignesh, Etienne, Sascha, Marie-Jean (anyone else) (location: restaurant le 19) \\
			\hline
		\end{tabular}
		
		Day 3:~ February 28 \\
		\vspace*{0.5em}
		\begin{tabular}{| p{4cm} p{13cm} |}
			\hline
			0900 & Arrival on campus \\
			0930 -- 1015 &  Vignesh Kannan \\
			1015 & Departure for Paris Gare de Lyon \\
			1156 & Train departs for Lausanne \\
			\hline
		\end{tabular}
	\end{center}
		
\end{document}