% !TeX spellcheck = en_us
\documentclass[a4paper,12pt,fleqn]{article}

\usepackage[left=1.50cm, right=1.50cm, top=4cm, bottom=2cm, 
headheight= 2.5cm, headsep= 4pt, footskip=25pt]{geometry}
\usepackage{fancyhdr}
\usepackage{lmodern}
\usepackage[table]{xcolor}
\usepackage{amssymb,amstext,amsmath}
\usepackage{booktabs}
\usepackage{bm}
\usepackage{setspace}
\usepackage{graphicx}
\usepackage[colorlinks]{hyperref}
\setlength{\parindent}{0mm}
\setlength{\parskip}{2mm}
\usepackage{enumitem}
\usepackage{titlesec}
\usepackage{gensymb}
\usepackage{pgfgantt}
\usepackage{lscape}
\usepackage[absolute]{textpos}
\usepackage{comment}
\usepackage{cite}
\usepackage{tikz}
\usepackage{pgfplots}
\usetikzlibrary{shapes.geometric}
\usepackage{array}
\newcolumntype{P}[1]{>{\centering\arraybackslash}p{#1}}
\usepackage{longtable}
\usepackage[font={small}]{caption}
\usepackage[most]{tcolorbox}
\usepackage{tikz}
\usepackage{stmaryrd}
\usepackage{wrapfig}
%\usepackage[none]{hyphenat}
\usepackage{ctable}
\usepackage{ragged2e}
\usepackage[official]{eurosym}
\newcommand*\circled[1]{\tikz[baseline=(char.base)]{
		\node[shape=circle,draw,inner sep=1pt] (char) {#1};}}
\usepackage[symbol]{footmisc}
\renewcommand{\thefootnote}{\fnsymbol{footnote}}
\renewcommand{\thempfootnote}{\fnsymbol{mpfootnote}}
\renewcommand*\footnoterule{}

\usepackage{xcolor}
\definecolor{docColor}{cmyk}{1.00	, 1.00  , 0   , 0.40}  % 0.90\% of black
\color{docColor}

\newtcolorbox{myhlbox}[2][]{colback=white, colframe=docColor, width=1.0\linewidth,
	boxrule=1.0pt, left=0.0pt, right=0.0pt, top=4pt, bottom=2.0pt,
	colbacktitle=docColor!20, coltitle=black, enhanced, attach boxed title to top left={xshift=3mm, yshift=-1mm},
	title=#2, #1}

\hypersetup{
	colorlinks=true,
	citecolor=blue,
	filecolor=magenta,
	linkcolor=blue,
	urlcolor=blue,
	pdftitle={SeminarAnnouncements},
	%bookmarks=true,,
}

\graphicspath{{./figs/}}

\definecolor{dgreen}{rgb}{0.0, 0.5, 0.0}

\allowdisplaybreaks

\newcommand*\mystrut[1]{\vrule width0pt height0pt depth#1\relax}
\renewcommand{\thesection}{\arabic{section}.}
\renewcommand{\thesubsection}{\arabic{section}.\arabic{subsection}}
\renewcommand{\thesubsubsection}{\arabic{section}.\arabic{subsection}.\arabic{subsubsection}}
\titleformat{\section}{\sffamily\large\bfseries}{\thesection}{1.0em}{}
\titleformat{\subsection}{\sffamily\large\bfseries}{\thesubsection}{1.0em}{}
\titleformat{\subsubsection}{\sffamily\normalsize\bfseries}{\thesubsubsection}{1.0em}{}
\titlespacing\section{0pt}{0pt plus 2pt minus 2pt}{0pt plus 2pt minus 2pt}
\renewcommand{\baselinestretch}{1.00}

\newcommand\CircledImage[1]{%
	\tikz\path[fill overzoom image={#1}, draw=docColor, line width=0mm]circle[radius=0.5\linewidth];%
}

\input{format/defs}

\fancypagestyle{titlepage}{
	\fancyhf{}
	\insertLXheader
}

\fancypagestyle{plain}{%
	\fancyhf{}
	\renewcommand{\headrulewidth}{0pt}
	\renewcommand{\footrulewidth}{0pt}
	\insertnormalpage
}

\pagestyle{plain}

\begin{document}
	\sffamily
\thispagestyle{titlepage}
\vspace*{-2em}
\begin{center}
	\huge \textbf{Mechanics Seminar series 2024 -- 25}
\end{center}

\begin{center}
	\Large 
	\textbf{Inertial cavitation in Soft Matter – Friend or Foe?} 
\end{center}
\begin{center}
	\large
	Christian Franck \\ 
	{\large Department of Mechanical Engineering \\
		University of Wisconsin-Madison
	} \\
	\vspace*{1em}
	\textbf{Date and Time}:~April 08, 2025 (2 -- 3 pm) \\
	\textbf{Venue}:~Amphi 104 (Pole Meca) 
\end{center}

\begin{myhlbox}{{\large Abstract}}
	The powerful and destructive nature of cavitation has long been appreciated. From cavitation-erosion on ship-based propellers, pumps, and impellers to the prey-stunning capability of the mantis shrimp, inertial cavitation is known to generate stresses on the order of gigapascals with internal bubble pressures and temperatures rivaling our sun. For soft matter systems in particular, understanding the large, high-rate deformation response of the material during cavitation has become paramount in being able to either mitigate or carefully harness its power in a plethora of engineering and clinical applications.
	 
	In this talk, I will present a two part lecture on our recent experimental developments in using inertial cavitation. In part one I will provide an overview of our previously developed  high to ultra-high strain rates ($10^{3} - 10^{8}~\text{s}^{-1}$) microrheology technique called Inertial Microcavitation Rheometry (IMR) along with an application for characterizing the finite deformation, viscoelastic material response of porcine brain tissue. In part two part of my talk, I will discuss the potential role of inertial cavitation in the injury signature and pathology of blast traumatic brain injuries and provide the first kind of quantification of critical injury thresholds for these types of cellular injuries. 
\end{myhlbox}

\vspace*{1em}

\begin{myhlbox}{{\large About the speaker}}
	\begin{wrapfigure}{r}{0.30\textwidth}
		\CircledImage{images/franck}
	\end{wrapfigure}
	\vspace*{0.5em}
	Christian Franck is the H.I. Romnes Faculty Fellow and Bjorn Borgen Professor in Mechanical Engineering at the University of Wisconsin-Madison. 
	He is the acting director of the Center for Traumatic Brain Injury at the University of Wisconsin-Madison and the ONR-funded "Physics-based Neutralization of Threats to Human Tissues and Organs" (PANTHER) program, which consists of over 30 PIs nationwide. Key objectives of Dr. Franck’s research program are in advanced detection and prevention of traumatic brain injuries by providing accelerated translation from basic science discovery to civilian and warfighter protection solutions. 
	\vspace*{2.5em}
\end{myhlbox}
\vfill
\begin{center}
	\Large
	\textbullet \hspace*{0.5em} \href{https://calendar.google.com/calendar/embed?src=lms.seminaires%40gmail.com&ctz=Europe%2FZurich}{Google calendar link} \hspace*{0.5em} \textbullet \hspace*{0.5em} \href{http://lms-seminars.polytechnique.fr/}{Seminar webpage link} \hspace*{0.5em} \textbullet
\end{center}
\end{document}