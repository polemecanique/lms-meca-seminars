% !TeX spellcheck = en_us
\documentclass[a4paper,12pt,fleqn]{article}

\usepackage[left=1.50cm, right=1.50cm, top=3cm, bottom=2cm, 
headheight= 2.5cm, headsep= 4pt, footskip=25pt]{geometry}
\usepackage{fancyhdr}
\usepackage{lmodern}
\usepackage[table]{xcolor}
\usepackage{amssymb,amstext,amsmath}
\usepackage{booktabs}
\usepackage{bm}
\usepackage{setspace}
\usepackage{graphicx}
\usepackage[colorlinks]{hyperref}
\setlength{\parindent}{0mm}
\setlength{\parskip}{2mm}
\usepackage{enumitem}
\usepackage{titlesec}
\usepackage{gensymb}
\usepackage{pgfgantt}
\usepackage{lscape}
\usepackage[absolute]{textpos}
\usepackage{comment}
\usepackage{cite}
\usepackage{tikz}
\usepackage{pgfplots}
\usetikzlibrary{shapes.geometric}
\usepackage{array}
\newcolumntype{P}[1]{>{\centering\arraybackslash}p{#1}}
\usepackage{longtable}
\usepackage[font={small}]{caption}
\usepackage[most]{tcolorbox}
\usepackage{tikz}
\usepackage{stmaryrd}
\usepackage{wrapfig}
%\usepackage[none]{hyphenat}
\usepackage{ctable}
\usepackage{ragged2e}
\usepackage[official]{eurosym}
\newcommand*\circled[1]{\tikz[baseline=(char.base)]{
		\node[shape=circle,draw,inner sep=1pt] (char) {#1};}}
\usepackage[symbol]{footmisc}
\renewcommand{\thefootnote}{\fnsymbol{footnote}}
\renewcommand{\thempfootnote}{\fnsymbol{mpfootnote}}
\renewcommand*\footnoterule{}

\usepackage{xcolor}
\definecolor{docColor}{cmyk}{1.00	, 1.00  , 0   , 0.40}  % 0.90\% of black
\color{docColor}

\newtcolorbox{myhlbox}[2][]{colback=white, colframe=docColor, width=1.0\linewidth,
	boxrule=1.0pt, left=0.0pt, right=0.0pt, top=4pt, bottom=2.0pt,
	colbacktitle=docColor!20, coltitle=black, enhanced, attach boxed title to top left={xshift=3mm, yshift=-1mm},
	title=#2, #1}

\hypersetup{
	colorlinks=true,
	citecolor=blue,
	filecolor=magenta,
	linkcolor=blue,
	urlcolor=blue,
	pdftitle={SeminarAnnouncements},
	%bookmarks=true,,
}

\graphicspath{{./figs/}}

\definecolor{dgreen}{rgb}{0.0, 0.5, 0.0}

\allowdisplaybreaks

\newcommand*\mystrut[1]{\vrule width0pt height0pt depth#1\relax}
\renewcommand{\thesection}{\arabic{section}.}
\renewcommand{\thesubsection}{\arabic{section}.\arabic{subsection}}
\renewcommand{\thesubsubsection}{\arabic{section}.\arabic{subsection}.\arabic{subsubsection}}
\titleformat{\section}{\sffamily\large\bfseries}{\thesection}{1.0em}{}
\titleformat{\subsection}{\sffamily\large\bfseries}{\thesubsection}{1.0em}{}
\titleformat{\subsubsection}{\sffamily\normalsize\bfseries}{\thesubsubsection}{1.0em}{}
\titlespacing\section{0pt}{0pt plus 2pt minus 2pt}{0pt plus 2pt minus 2pt}
\renewcommand{\baselinestretch}{1.00}

\newcommand\CircledImage[1]{%
	\tikz\path[fill overzoom image={#1}, draw=docColor, line width=0mm]circle[radius=0.5\linewidth];%
}

\input{format/defs}

\fancypagestyle{titlepage}{
	\fancyhf{}
	\insertLXheader
}

\fancypagestyle{plain}{%
	\fancyhf{}
	\renewcommand{\headrulewidth}{0pt}
	\renewcommand{\footrulewidth}{0pt}
	\insertnormalpage
}

\pagestyle{plain}

\begin{document}
	\sffamily
\thispagestyle{titlepage}
\vspace*{-3em}
\begin{center}
	\huge \textbf{LMS Seminar}
\end{center}
\begin{center}
	\Large 
	\textbf{Adjoint variable method and applications: from design optimization to force computation} 
\end{center}
\begin{center}
	\Large
	Thèodore Cherrière \\
	{\large GeePs laboratory \\ 
	Centrale Supèlec} \\
	\vspace*{1em}
	{\Large
	\textbf{Date and Time}:~September 04, 2025 (2 -- 3 pm)
	
	\textbf{Venue}:~Amphi 104 (Pole Meca) }
\end{center}
\vspace*{-1em}
\begin{myhlbox}{{\large Abstract}}
	The adjoint method is a very effective technique for calculating derivatives with respect to many variables, accounting for implicit relations. In this talk, basic principles of the method and its implementation will be discussed, together with various applications, from topology optimization to theoretical connections with the calculation of magnetic forces using the virtual work method.
\end{myhlbox}

\vspace*{1em}

\begin{myhlbox}{{\large About the speaker}}
	\begin{wrapfigure}{r}{0.30\textwidth}
		\vspace*{0em}
		\CircledImage{images/cherriere}
	\end{wrapfigure}
	Théodore Cherrière obtained his PhD from École Normale Supérieure Paris-Saclay – Université Paris-Saclay on the magneto-mechanical topology optimization of electrical machines. For this work, he has received prizes such as the Paul Caseau prize delivered by EDF and Académie des Technologies, and the Ampère prize delivered by SEE (“Société de l'électricité, de l'électronique et des technologies de l'information et de la communication”). On the same topic, he did a post-doc at the Johann Radon Institute for Computational and Applied Mathematics in Linz, Austria, developing a multi-material topology optimization framework. He is now an assistant professor (maître de conferences) of electrical engineering at CentraleSupélec – Université Paris-Saclay. His research interests are in computational design optimization methods, which are dedicated primarily to electromagnetic actuators.
\end{myhlbox}
\end{document}